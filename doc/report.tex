\documentclass{article}
\usepackage{amsmath}
\usepackage{amsfonts}
\usepackage{graphicx}
\usepackage{placeins}
\usepackage[top=1in]{geometry}
\usepackage{minted}
\usepackage{hyperref}

\author{Jaakko Hannikainen - Compilers}
\title{Miniplc, an interpreter for the MiniPL programming language}

\begin{document}
\maketitle

\noindent
\begin{minipage}{0.49\textwidth}
\begin{minted}{python}
var nTimes : int := 0;
print "How many times? ";
read nTimes;
var x : int;
for x in 0..nTimes-1 do
    print x;
    print " : Hello, World!\n";
end for;
assert (x = nTimes);
\end{minted}
\end{minipage}
\begin{minipage}{0.49\textwidth}
\begin{minted}{text}
$ java -jar build/libs/miniplc.jar loopExample.mpl
How many times? 5
0 : Hello, World!
1 : Hello, World!
2 : Hello, World!
3 : Hello, World!
4 : Hello, World!
$
\end{minted}
\end{minipage}

\vfill

\section{Introduction}
Miniplc is an interpreter for the MiniPL programming language. \\[1.5em]

\noindent
Basic usage:
\begin{minted}{shell}
    ./gradlew jar
    java -jar build/libs/miniplc.jar helloWorld.mpl
\end{minted}

\noindent
Running tests:
\begin{minted}{shell}
    ./gradlew test # normal tests
    ./gradlew pitest # mutation tests
\end{minted}

\vfill

\newpage
\section{Architecture}
\label{sec:architecture}
The interpreter has a simple multi-pass design: first the program is tokenized,
and then resulting token stream is parsed using a simple grammar. The resulting
tree is then executed.

The tokenizer (fi.jgke.miniplc.tokenizer.Tokenizer) is a simple ad-hoc
tokenizer - As long as the input stream is not empty, the tokenizer reads a
character at a time until it has formed a valid token. In some cases, the
tokenizer peeks one character, for example when differentiating between
division '/' and a single-line comment '//'.
As pseudocode, the tokenizer works as follows:
\begin{minted}{text}
read one character from input

if character is one of ['!', '&', '(', ')', '*', '+', '-', ';', '<', '=']
    return character

switch (character) {
    case '/':
        read second character from input
        if second character is '/' read input until newline and parse new token
        else push character back to stream and return '/'

    case ':':
        read second character from input
        if second character is '=' return ':='
        else push character back to stream and return ':'

    case '.':
        read second character from input
        if second character is '.' return '..'
        else throw UnexpectedCharacterException

    case '"':
        read characters until '"' and preceding character is not '\'
        return string
}

if character is a number:
    read characters until next character is not a number
    return number

if character is a letter:
    read characters until next character is not a letter or a number
    if string is a keyword:
        return string as keyword
    return string as identifier

throw UnexpectedCharacterException
\end{minted}
\newpage

After tokenizing, the tokens are returned to the Executor
(fi.jgke.miniplc.interpreter.Executor), which gives the resulting stream to the
tree parser, located in the package fi.jgke.miniplc.builder. The parser is
trivially modifiable to enable parsing LL(k) grammars, but in the case of this
language, only LL(1) features are used.

The syntax of the language is handled using a simple Java-based DSL, located in
fi.jgke.miniplc.builder.Syntax.

\begin{minted}{Java}
/*
 * <operand> ::=  <intConstant> | <stringConstant> | <boolConstant>
 *            |   <identifier>
 *            |   "(" expr ")"
 */
public static Rule operand() {
    return lazy(() -> any(
            rule(OperandHandlers::handleConstant,
                 any(IntConst, StringConst, BoolConst)),
            rule(OperandHandlers::handleIdentifier,
                 Identifier),
            rule(OperandHandlers::handleExpression,
                 OpenBrace, expression(), CloseBrace)
    ));
}
\end{minted}

The above example is the parse rule for operands. The rule states that an
operand is a token stream which either

\begin{itemize}
    \item is any of: IntConst (eg. 5), StringConst (eg. "foo") or BoolConst
        (eg. true),
    \item is an identifier (any string sequence which is not a keyword) or
    \item starts with a open brace '(' and contains an expression which is
        followed by a close brace ')'.
\end{itemize}
\noindent
If none of these is true, an exception is thrown. \\

The tree is then parsed: if the topmost token is any constant or identifier,
it is taken off the token stream and passed to the handler - either handleConstant or handleIdentifier.
Otherwise, an expression is parsed, and passed to the handler.

\FloatBarrier
\begin{figure}[ht!]
    \begin{center}
        \makebox[\textwidth]{
            \includegraphics[width=\paperwidth]{diagram}}
    \end{center}
    \caption{Simplified UML-like diagram, generated using IntelliJ IDEA.}
    \label{fig:uml}
\end{figure}
\FloatBarrier

The control flow is simple: First the program is passed (as a String) to
the Tokenizer through creating a new TokenQueue. When the parsing is done, the
list of tokens is passed to the topmost rule in Builder: program(). There the
recursive-descent parser passes the queue around to different rules. When rule
consumes tokens from a token queue, it converts to a ConsumedRule, which are
handled by the parent rules.

Alongside the token queue, a Context is passed, which handles currently
available variables, variable scoping and input/output.

\section{Testing}
The interpreter has a comprehensive automatic test suite. It testes various
aspects of the interpreter, ranging from unit tests testing single functions to
running sample programs. Tests are implemented usin Junit 4, Mockito and
Hamcrest. Many of the tests mock Input and Output in order to test read and
print functionalities. The tests are organized under ./src/test. The packages
are as follows:

\begin{description}
    \item[fi.jgke.miniplc.builder]
        Tests for the recursive-descent parser.

    \item[fi.jgke.miniplc.language]
        Tests for tokens and both unary and binary operations.

    \item[fi.jgke.miniplc.misc]
        Miscellaneous tests that needlessly increase test coverage.

    \item[fi.jgke.miniplc.samples]
        A collection of samples, which test the whole interpreter at once.

    \item[fi.jgke.miniplc.tokenizer]
        Tests for the tokenizer.

    \item[fi.jgke.miniplc.unit]
        Various unit tests.
\end{description}

\newpage

\section{Tokens patterns as regular expressions}

\begin{minted}{text}
# Tokens which can be directly matched
simple-tokens = ! | & | \( | \) | \* | \+ | - | ; | < | =

# one line comment and input until line end
comment = //.*\n

# division operator
divide = /

# assignment operator
assign = :=

# colon token
colon = :

# range token
range = \.\.

# string constant
string = ".*"

# number constant
number = [0-9]+

# identifier, keyword or a boolean
identifier-or-keyword = [a-zA-Z][a-zA-Z0-9]*

# white space
white-space = [ \t\n]

\end{minted}


\newpage

\section{LL(1) Syntax}

\begin{minted}[mathescape, escapeinside=&&]{text}
 <prog>   ::=  <stmt> ";" <stmts>

 <stmts>  ::=  <stmt> ";" <stmts>
           |   &$\varepsilon$&

 <stmt>   ::=  "var" <var_ident> ":" <type> <maybe_assign>
           |   "for" <var_ident> "in" <expr> ".." <expr> "do"
                  <stmts> "end" "for"
           |   "read" <var_ident>
           |   "print" <expr>
           |   "assert" "(" <expr> ")"
           |   <var_ident> ":=" <expr>

 <maybe_assign> ::= ":=" <expr>
                 |   &$\varepsilon$&

 <expr>   ::= <unary_op> <opnd>
            | <opnd> <maybe_operand>

 <maybe_operand> ::= <binary_op> <opnd>
                  |  &$\varepsilon$&

 <opnd>   ::=  <int>
           |   <string>
           |   <bool>
           |   <var_ident>
           |   "(" expr ")"

 <type>   ::=  "int" | "string" | "bool"
 <var_ident> ::= <ident>

 <reserved keyword> ::=
              "var" | "for" | "end" | "in" | "do" | "read" |
              "print" | "int" | "string" | "bool" | "assert"
\end{minted}

\newpage

\section{Syntax as represented in the source code}
\begin{minted}{java}
/*
 * <program> ::= <statement> ";" <statements>
 */
public static Rule program() {
    return lazy(() -> rule(StatementsHandlers::executeStatements,
                           statement(), Semicolon, statements())
    );
}

/*
 * <statements> ::= <statement> ";" <statements>
 *               |  epsilon
 */
public static Rule statements() {
    return lazy(() -> any(
            rule(StatementsHandlers::executeStatements,
                 statement(), Semicolon, statements()),
            empty()
    ));
}

/*
 * <statement> ::=  "var" <identifier> ":" <type> <maybe_assign>
 *              |   <identifier> ":=" <expression>
 *              |   "for" <identifier> "in" <expression> ".." <expression> "do"
 *                     <statements> "end" "for"
 *              |   "read" <identifier>
 *              |   "print" <expression>
 *              |   "assert" "(" <expression> ")"
 */
public static Rule statement() {
    return lazy(() -> any(
            rule(StatementHandlers::createVariable,
                 Var, Identifier, Colon, Type, maybe_assign()),
            rule(StatementHandlers::updateVariable,
                 Identifier, Assign, expression()),
            rule(StatementHandlers::printExpression,
                 Print, expression()),
            rule(StatementHandlers::readVariable,
                 Read, Identifier),
            rule(StatementHandlers::assertExpression,
                 Assert, OpenBrace, expression(), CloseBrace),
            rule(StatementHandlers::forLoop,
                 For, Identifier, In, expression(), Range, expression(), Do,
                     statements(),
                 End, For)
    ));
}


/*
 * <maybe_assign> ::= ":=" <expression>
 *                 |  epsilon
 */
private static Rule maybe_assign() {
    return lazy(() -> any(
            all(Assign, expression()),
            empty()
    ));
}

/*
 * <expression> ::= <unaryOperator> <operand>
 *               | <operand> <maybe_operand>
 */
public static Rule expression() {
    return lazy(() -> any(
            rule(ExpressionHandlers::handleNot,
                 Not, operand()),
            rule(ExpressionHandlers::handleOperation,
                 operand(), maybe_operand())
    ));
}

/*
 * <maybe_operand> ::= <binaryOperator> <operand>
 *                  |  epsilon
 */
private static Rule maybe_operand() {
    return lazy(() -> any(
            all(operator(), operand()),
            empty()
    ));
}

/*
 * operator ::= "+" | "-" | "*" | "/" | "<" | "=" | "\&"
 */
public static Rule operator() {
    return any(Plus, Minus, Times, Divide, LessThan, Equals, And);
}











/*
 * <operand> ::=  <intConstant> | <stringConstant> | <boolConstant>
 *            |   <identifier>
 *            |   "(" expr ")"
 */
public static Rule operand() {
    return lazy(() -> any(
            rule(OperandHandlers::handleConstant,
                 any(IntConst, StringConst, BoolConst)),
            rule(OperandHandlers::handleIdentifier,
                 Identifier),
            rule(OperandHandlers::handleExpression,
                 OpenBrace, expression(), CloseBrace)
    ));
}
\end{minted}

\section{Interpreter details}
The error messages could use some more work. Currently the interpreter reports
the line for the syntax error, but pointing at the exact token or printing the
line in question could help. Adding or changing tokens to the expected versions
could allow printing all errors in the file, rather than abandoning parsing at
the first error. Currently error messages look like the following example: \\[0.5em]
errorExample.mpl:
\begin{minted}[mathescape, escapeinside=&&]{python}
    print 5;
    print 6;
    print 5 foo;
    print 7;
    print 8;
\end{minted}
\begin{minted}[mathescape, escapeinside=&&]{text}
    $ java -jar build/libs/miniplc.jar errorExample.mpl
    Unexpected token near line 3: IDENTIFIER (expected SEMICOLON)
    $
\end{minted}

The internal representation is built using a simple DSL, located in
fi.jgke.miniplc.builder.Syntax. It is explained in more detail in
chapter~\ref{sec:architecture}. When the token stream is parsed, the DSL is
used to convert the token stream to a tree structure, where each node either
is a single token (such as a constant) or a complex statement (eg. for loop).
The parsed tree is then executed.

Error handling is very simple - when the interpreter reaches some invalid
state, such as reading an unexpected token, an exception is thrown. The
exception then falls through the entire interpreter and is caught at the main
function.

\newpage
\appendix
\section{Original project definition}

This is a copy of the original project definition located at
\url{https://www.cs.helsinki.fi/u/vihavain/k16/Compilers/project/miniplsyntax\_2016.html}

\subsection{Syntax and semantics of Mini-PL (9.2.2016)}
Mini-PL is a simple programming language designed for pedagogic purposes. The
language is purposely small and is not actually meant for any real programming.
Mini-PL contains few statements, arithmetic expressions, and some IO
primitives. The language uses static typing and has three built-in types
representing primitive values: int, string, and bool. The BNF-style syntax of
Mini-PL is given below, and the following paragraphs informally describe the
semantics of the language.

Mini-PL uses a single global scope for all different kinds of names. All
variables must be declared before use, and each identifier may be declared once
only. If not explicitly initialized, variables are assigned an appropriate
default value.

The Mini-PL read statement can read either an integer value or a single word
(string) from the input stream. Both types of items are whitespace-limited (by
blanks, newlines, etc). Likewise, the print statement can write out either
integers or string values. A Mini-PL program uses default input and output
channels defined by its environment. Additionally, Mini-PL includes an assert
statement that can be used to verify assertions (assumptions) about the state
of the program. An assert statement takes a bool argument. If an assertion
fails (the argument is false) the system prints out a diagnostic message.  The
arithmetic operator symbols '+', '-', '*','/' represent the following
functions:

\begin{minted}{text}
  "+" : (int, int) -> int            // integer addition
  "-" : (int, int) -> int            // integer subtraction
  "*" : (int, int) -> int            // integer multiplication
  "/" : (int, int) -> int            // integer division
\end{minted}
The operator '+' also represents string concatenation (i.e., this operator
symbol is overloaded):

\begin{minted}{text}
  "+" : (string, string) -> string   // string concatenation
\end{minted}
The operators '\&' and '!' represent logical operations:

\begin{minted}{text}
  "\&" : (bool, bool) -> bool         // logical and
  "!" : (bool) -> bool                // logical not
\end{minted}
The operators '=' and b '<' are overloaded to represent the comparisons between
two values of the same type T (int, string, or bool):

\begin{minted}{text}
  "=" : (T, T) -> bool               // equality comparison
  "<" : (T, T) -> bool               // less-than comparison
\end{minted}
A for statement iterates over the consequent values from a specified integer
range. The expressions specifying the beginning and end of the range are
evaluated once only (at the beginning of the for statement). The for control
variable behaves like a constant inside the loop: it cannot be assigned another
value (before exiting the for statement). A control variable needs to be
declared before its use in the for statement (in the global scope). Note that
loop control variables are not declared inside for statements.

\subsection{Context-free grammar for Mini-PL}
The syntax definition is given in so-called Extended Backus-Naur form (EBNF).
In the following Mini-PL grammar, the notation X* means 0, 1, or more
repetitions of the item X. The '|' operator is used to define alternative
constructs. Parentheses may be used to group together a sequence of related
symbols. Brackets ("[" "]") may be used to enclose optional parts (i.e., zero
or one occurrence). Reserved keywords are marked bold (as "var"). Operators,
separators, and other single or multiple character tokens are enclosed within
quotes (as: ".."). Note that nested expressions are always fully parenthesized
to specify the execution order of operations.

\begin{minted}{text}
 <prog>   ::=  <stmts>
 <stmts>  ::=  <stmt> ";" ( <stmt> ";" )*
 <stmt>   ::=  "var" <var_ident> ":" <type> [ ":=" <expr> ]
           |   <var_ident> ":=" <expr>
           |   "for" <var_ident> "in" <expr> ".." <expr> "do"
                  <stmts> "end" "for"
           |   "read" <var_ident>
           |   "print" <expr>
           |   "assert" "(" <expr> ")"

 <expr>   ::=  <opnd> <op> <opnd>
           |   [ <unary_op> ] <opnd>

 <opnd>   ::=  <int>
           |   <string>
           |   <var_ident>
           |   "(" expr ")"

 <type>   ::=  "int" | "string" | "bool"
 <var_ident> ::= <ident>

 <reserved keyword> ::=
              "var" | "for" | "end" | "in" | "do" | "read" |
              "print" | "int" | "string" | "bool" | "assert"
\end{minted}
\subsection{Lexical elements}

In the syntax definition the symbol <ident> stands for an identifier (name). An
identifier is a sequence of letters, digits, and underscores, starting with a
letter. Uppercase letters are distinguished from lowercase.

In the syntax definition the symbol <int> stands for an integer constant
(literal). An integer constant is a sequence of decimal digits. The symbol
<string> stands for a string literal. String literals follow the C-style
convention: any special characters, such as the quote character (") or
backslash (\\), are represented using escape characters (e.g.: \\").

A limited set of operators include (only!) the ones listed below.

\begin{minted}{text}
'+' | '-' | '*' | '/' | '<' | '=' | '\&' | '!'
\end{minted}

In the syntax definition the symbol <op> stands for a binary operator symbol.
There is one unary operator symbol (<unary\_op>): '!', meaning the logical not
operation. The operator symbol '\&' stands for the logical and operation. Note
that in Mini-PL, '=' is the equal operator - not assignment.

The predefined type names (e.g.,"int") are reserved keywords, so they cannot be
used as (arbitrary) identifiers. In a Mini-PL program, a comment may appear
between any two tokens. There are two forms of comments: one starts with "/*",
ends with "*/", can extend over multiple lines, and may be nested. The other
comment alternative begins with "//" and goes only to the end of the line.

\subsection{Sample programs}
\begin{minted}{text}
     var X : int := 4 + (6 * 2);
     print X;

     var nTimes : int := 0;
     print "How many times?";
     read nTimes;
     var x : int;
     for x in 0..nTimes-1 do
         print x;
         print " : Hello, World!\n";
     end for;
     assert (x = nTimes);

     print "Give a number";
     var n : int;
     read n;
     var v : int := 1;
     var i : int;
     for i in 1..n do
         v := v * i;
     end for;
     print "The result is: ";
     print v;
\end{minted}

\end{document}
